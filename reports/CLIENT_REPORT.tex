% Options for packages loaded elsewhere
\PassOptionsToPackage{unicode}{hyperref}
\PassOptionsToPackage{hyphens}{url}
\documentclass[
]{article}
\usepackage{xcolor}
\usepackage[margin=1in]{geometry}
\usepackage{amsmath,amssymb}
\setcounter{secnumdepth}{-\maxdimen} % remove section numbering
\usepackage{iftex}
\ifPDFTeX
  \usepackage[T1]{fontenc}
  \usepackage[utf8]{inputenc}
  \usepackage{textcomp} % provide euro and other symbols
\else % if luatex or xetex
  \usepackage{unicode-math} % this also loads fontspec
  \defaultfontfeatures{Scale=MatchLowercase}
  \defaultfontfeatures[\rmfamily]{Ligatures=TeX,Scale=1}
\fi
\usepackage{lmodern}
\ifPDFTeX\else
  % xetex/luatex font selection
\fi
% Use upquote if available, for straight quotes in verbatim environments
\IfFileExists{upquote.sty}{\usepackage{upquote}}{}
\IfFileExists{microtype.sty}{% use microtype if available
  \usepackage[]{microtype}
  \UseMicrotypeSet[protrusion]{basicmath} % disable protrusion for tt fonts
}{}
\makeatletter
\@ifundefined{KOMAClassName}{% if non-KOMA class
  \IfFileExists{parskip.sty}{%
    \usepackage{parskip}
  }{% else
    \setlength{\parindent}{0pt}
    \setlength{\parskip}{6pt plus 2pt minus 1pt}}
}{% if KOMA class
  \KOMAoptions{parskip=half}}
\makeatother
\usepackage{longtable,booktabs,array}
\usepackage{calc} % for calculating minipage widths
% Correct order of tables after \paragraph or \subparagraph
\usepackage{etoolbox}
\makeatletter
\patchcmd\longtable{\par}{\if@noskipsec\mbox{}\fi\par}{}{}
\makeatother
% Allow footnotes in longtable head/foot
\IfFileExists{footnotehyper.sty}{\usepackage{footnotehyper}}{\usepackage{footnote}}
\makesavenoteenv{longtable}
\usepackage{graphicx}
\makeatletter
\newsavebox\pandoc@box
\newcommand*\pandocbounded[1]{% scales image to fit in text height/width
  \sbox\pandoc@box{#1}%
  \Gscale@div\@tempa{\textheight}{\dimexpr\ht\pandoc@box+\dp\pandoc@box\relax}%
  \Gscale@div\@tempb{\linewidth}{\wd\pandoc@box}%
  \ifdim\@tempb\p@<\@tempa\p@\let\@tempa\@tempb\fi% select the smaller of both
  \ifdim\@tempa\p@<\p@\scalebox{\@tempa}{\usebox\pandoc@box}%
  \else\usebox{\pandoc@box}%
  \fi%
}
% Set default figure placement to htbp
\def\fps@figure{htbp}
\makeatother
\setlength{\emergencystretch}{3em} % prevent overfull lines
\providecommand{\tightlist}{%
  \setlength{\itemsep}{0pt}\setlength{\parskip}{0pt}}
\usepackage{bookmark}
\IfFileExists{xurl.sty}{\usepackage{xurl}}{} % add URL line breaks if available
\urlstyle{same}
\hypersetup{
  pdftitle={Client report},
  hidelinks,
  pdfcreator={LaTeX via pandoc}}

\title{Client report}
\author{}
\date{\vspace{-2.5em}}

\begin{document}
\maketitle

\section{Introduction}\label{introduction}

This brief report assesses whether the dataset supports a clear and
defensible message that could be communicated to a general audience. The
focus is on a small number of interpretable comparisons, with explicit
limitations and a practical next step.

\section{Client brief}\label{client-brief}

\begin{quote}
We would like you to assess whether our dataset supports a clear and
defensible message that could be communicated to a general audience. In
practical terms, we want to know whether the data show a pattern that is
relevant to our decision-making, sufficiently consistent to justify a
public-facing narrative, and unlikely to be explained by obvious
artefacts such as differences in the make-up of the groups, missing or
incomplete data, or changes in definitions over time.

Please provide an initial evidence-led summary that (i) states the most
defensible claim the data can support at this stage, (ii) sets out the
key evidence for that claim, (iii) explains the main uncertainties and
limitations and their likely impact, and (iv) recommends the most useful
next step to strengthen confidence in the story.
\end{quote}

\section{Data description}\label{data-description}

The data file used was: data/raw/week5\_dataset.csv.

\textbf{Note:} this file contains repeated monthly observations. To keep
the deliverable structure (one record per customer), this report
analyses the most recent month in the file: \textbf{2025-12}. A
month-by-month stability check is recommended in the final section.

Each record in this report represents a single customer in the analysed
extract. Customers are split into two groups (Control and Treatment)
intended to represent whether the customer received a marketing
intervention.

\section{Data description}\label{data-description-1}

This report uses the field definitions provided by the client (see
\texttt{DATA\_DICTIONARY.md}). Where definitions are provisional, we
flag the most decision-relevant uncertainties in the limitations
section.

\section{Summary of data
characteristics}\label{summary-of-data-characteristics}

\begin{itemize}
\tightlist
\item
  Records analysed: 129
\item
  Group counts: Control=73, Treatment=56
\item
  Missing outcome (made\_purchase\_30d): 0.0\%
\end{itemize}

\subsection{Table 1: baseline characteristics by
group}\label{table-1-baseline-characteristics-by-group}

\begin{longtable}[]{@{}
  >{\raggedright\arraybackslash}p{(\linewidth - 6\tabcolsep) * \real{0.3143}}
  >{\raggedright\arraybackslash}p{(\linewidth - 6\tabcolsep) * \real{0.1429}}
  >{\raggedright\arraybackslash}p{(\linewidth - 6\tabcolsep) * \real{0.4143}}
  >{\raggedright\arraybackslash}p{(\linewidth - 6\tabcolsep) * \real{0.1286}}@{}}
\caption{Table 1. Baseline summary by group.}\tabularnewline
\toprule\noalign{}
\begin{minipage}[b]{\linewidth}\raggedright
variable
\end{minipage} & \begin{minipage}[b]{\linewidth}\raggedright
group
\end{minipage} & \begin{minipage}[b]{\linewidth}\raggedright
summary
\end{minipage} & \begin{minipage}[b]{\linewidth}\raggedright
level
\end{minipage} \\
\midrule\noalign{}
\endfirsthead
\toprule\noalign{}
\begin{minipage}[b]{\linewidth}\raggedright
variable
\end{minipage} & \begin{minipage}[b]{\linewidth}\raggedright
group
\end{minipage} & \begin{minipage}[b]{\linewidth}\raggedright
summary
\end{minipage} & \begin{minipage}[b]{\linewidth}\raggedright
level
\end{minipage} \\
\midrule\noalign{}
\endhead
\bottomrule\noalign{}
\endlastfoot
customer\_age & Control & Median 41.0 (IQR 17.0), n=73 & NA \\
customer\_age & Treatment & Median 43.0 (IQR 17.5), n=56 & NA \\
avg\_order\_value\_gbp & Control & Median 44.9 (IQR 9.6), n=73 & NA \\
avg\_order\_value\_gbp & Treatment & Median 46.7 (IQR 9.6), n=56 & NA \\
orders\_last\_90d & Control & Median 3.0 (IQR 2.0), n=73 & NA \\
orders\_last\_90d & Treatment & Median 3.0 (IQR 3.0), n=56 & NA \\
web\_sessions\_last\_30d & Control & Median 9.0 (IQR 4.0), n=73 & NA \\
web\_sessions\_last\_30d & Treatment & Median 7.0 (IQR 4.2), n=56 &
NA \\
area\_affluence\_score & Control & Median 27.0 (IQR 9.0), n=73 & NA \\
area\_affluence\_score & Treatment & Median 30.0 (IQR 10.2), n=56 &
NA \\
customer\_type & Control & 31 (42.5\%) & New \\
customer\_type & Control & 42 (57.5\%) & Existing \\
customer\_type & Treatment & 19 (33.9\%) & New \\
customer\_type & Treatment & 37 (66.1\%) & Existing \\
discount\_eligible & Control & 41 (56.2\%) & No \\
discount\_eligible & Control & 32 (43.8\%) & Yes \\
discount\_eligible & Treatment & 23 (41.1\%) & No \\
discount\_eligible & Treatment & 33 (58.9\%) & Yes \\
made\_purchase\_30d & Control & 46 (63.0\%) & No \\
made\_purchase\_30d & Control & 27 (37.0\%) & Yes \\
made\_purchase\_30d & Treatment & 28 (50.0\%) & No \\
made\_purchase\_30d & Treatment & 28 (50.0\%) & Yes \\
\end{longtable}

\{Yadavan\}This is a Table 1, wow!

\textbf{Interpretation.} Table 1 summarises how similar the Control and
Treatment groups look on key observable characteristics. Large
differences here can explain apparent outcome differences without
requiring a true intervention effect.

\section{Claim}\label{claim}

In this extract, the Treatment group shows a higher 30-day purchase rate
than Control (difference: 13.0 percentage points).

\section{Data and approach}\label{data-and-approach}

We compare outcomes between Treatment and Control using simple
descriptive summaries. The goal is not to prove causality, but to
identify whether there is a pattern that is large enough and
sufficiently credible to justify a cautious public-facing narrative, and
to identify the single most useful next step to strengthen confidence.

\section{Evidence}\label{evidence}

\subsection{Visualisation 1: purchase rate by
group}\label{visualisation-1-purchase-rate-by-group}

\pandocbounded{\includegraphics[keepaspectratio]{CLIENT_REPORT_files/figure-latex/fig1-1.pdf}}

\textbf{Interpretation.} This chart shows the percentage of customers
who purchased within 30 days, for Treatment vs Control. The error bars
reflect uncertainty from the sample size and help indicate whether the
difference is likely to be due to random variation.

\subsection{Visualisation 2: average order value by
group}\label{visualisation-2-average-order-value-by-group}

\pandocbounded{\includegraphics[keepaspectratio]{CLIENT_REPORT_files/figure-latex/fig2-1.pdf}}

\textbf{Interpretation.} This chart summarises whether one group tends
to have higher typical spend than the other. Differences here can matter
for interpretation, because higher-value customers may respond
differently regardless of the intervention.

\section{Uncertainty and limitations}\label{uncertainty-and-limitations}

\begin{enumerate}
\def\labelenumi{\arabic{enumi}.}
\item
  \textbf{Group differences may reflect who was selected, not the
  intervention.} If Treatment and Control differ in key baseline
  characteristics (Table 1), then outcome differences may be explained
  by selection rather than the intervention itself. This would weaken
  the credibility of a public-facing claim of impact.
\item
  \textbf{Outcome timing and definitions are provisional.} The
  definitions (for example how ``30 days'' is anchored, and whether
  purchases are counted when placed or fulfilled) could change the
  measured purchase rate. If the definition shifted, the story could
  change materially.
\item
  \textbf{If the source file is time-series (Week 5), this report uses
  only the most recent month.} The claim may not hold consistently over
  time. If seasonality or a recording change is present, a single-month
  snapshot may be misleading.
\end{enumerate}

\section{Recommendation and next
steps}\label{recommendation-and-next-steps}

The most useful next step is to confirm the intervention definition and
eligibility rules (including anchor dates), then repeat the comparison
with one simple robustness check: stratify the purchase rate by
\textbf{customer\_type} (New vs Existing) to see whether the pattern
holds within comparable customers. If this is a time-series dataset,
extend the analysis to month-by-month purchase rates to check stability
across time and rule out seasonal or definition-driven artefacts.

\end{document}
